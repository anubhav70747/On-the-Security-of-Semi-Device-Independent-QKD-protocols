\appendix
\section*{Appendix}
\subsection*{(i)}
W.l.o.g. we can consider Eve's projective measurements to be, 
\begin{eqnarray}
&& |M_e^{E=0}\rangle=\cos\frac{\alpha_e}{2}|0\rangle+\sin\frac{\alpha_e}{2}|1\rangle \nonumber \\
&& |M_e^{E=1}\rangle = |M_e^{E=0}\rangle^\perp \label{a4},
\end{eqnarray}
and $M_e^{E=0}=|M_e^{E=0}\rangle \langle M_e^{E=0} |$,$M_e^{E=0}=|M_e^{E=1}\rangle \langle M_e^{E=1}|$. Now one can rewrite (\ref{max2}) as,
\be \label{coolway!}
P_E^{max}(\eta) = \frac{1}{2}\max\left\{P_E^{a_0=a_1} + P_E^{a_0\neq a_1}(\eta) \right\},
\ee
where
\begin{equation} \label{subset1}
\begin{split}
P_E^{a_0 = a_1}=&
\frac{1}{4} Tr\bigg(\rho_{000}M^0_0 + \rho_{001}M^0_1 + \\
& \rho_{110}M^1_0 + \rho_{111}M^1_1\bigg), \\ 
\end{split}
\end{equation}
\begin{equation} \label{subset2}
\begin{split}
P_E^{a_0 \neq a_1}(\eta)= & \frac{1}{4} Tr\bigg(\rho_{010}\frac{M^0_0+\eta M^1_0}{1+\eta} + \\ & \rho_{011}\frac{M^1_1+\eta M^0_1}{1+\eta} + \rho_{100}\frac{M^1_0+\eta M^0_0}{1+\eta} + \\ & \rho_{101}\frac{M^1_1+\eta M^0_1}{1+\eta}\bigg.).
\end{split}
\end{equation}
Notice that (\ref{subset1}) and (\ref{subset2}) divide Alice's preparations into two mutually exclusive subsets. Alice's states that maximize $P_E^{a_0=a_1}$, $\rho_{0,0,0},\rho_{0,0,1},\rho_{1,1,0},\rho_{1,1,1}$ remain the same irrespective of whether Eve was able to correctly guess Bob's input ($e=b$) or not ($e\neq b$) simply because both of Alice's input are the same. This allows Eve to set these states equivalent to the projectors $M_0^0,M_1^0,M_0^1,M_1^1,$ respectively.   Which in-turn allows one to re-write (\ref{coolway!}) as,
\begin{equation}
\begin{split} \label{labelme}
P_E^{max}(\eta) = \frac{1}{2}(1+\max\{P_E^{a_0\neq a_1}(\eta)).
\end{split}
\end{equation}
Now in-order to find the maximum value of (\ref{subset2}) consider one of the terms involved,
\begin{equation}
\begin{split}
Tr\bigg(\rho_{010}\frac{M_0^0 + \eta M_0^1}{1+\eta}\bigg) = \frac{1}{1+\eta} - \frac{1-\eta}{1+\eta}Tr\bigg(\rho_{010}M_0^1\bigg),
\end{split}
\end{equation}
where the equality stems from the fact that $M_0^0=\mathbb{I}-M_0^1$.
The maximum for this term is reached by setting $\rho_{010}=M_0^0$ which yields the final security condition (\ref{maxMe}).
\subsection*{(ii)}
W.l.o.g she fixes Alice's preparations to be MUBs,
\begin{eqnarray}
&& \rho_{00}=|0\rangle\langle0|, \nonumber \\
&& \rho_{01}=\frac{1}{2}(|0\rangle+|1\rangle)(\langle0|+\langle1|), \nonumber \\
&& \rho_{10}=\frac{1}{2}(|0\rangle-|1\rangle)(\langle0|-\langle1|), \nonumber \\
&& \rho_{11}=|1\rangle\langle 1|, \label{c}
\end{eqnarray}
which is an optimal set of states for the standard ($2\to1$) QRAC \cite{Am2}. Next w.l.o.g. we characterize Eve's projective measurement using vectors from the same plane as the states in (\ref{c}),
\begin{eqnarray}
&& |M_e^{E=0}\rangle=\cos\frac{\alpha_e}{2}|0\rangle+\sin\frac{\alpha_e}{2}|1\rangle, \nonumber \\
&& |M_e^{E=1}\rangle = |M_e^{E=0}\rangle^\perp \label{a4},
\end{eqnarray}
and $M_e^{E}=|M_e^E\rangle\langle M_e^E|$. This allows us to partition (\ref{max2}) into two parts based on different values of $e$. These parts are independent and, due to symmetry, equal. Therefore we may re-write $P_E(\eta)$ as ,
\be
P_E(\eta)=\max\left\{\frac{P_{E_0}^0+\eta P_{E_1}^0}{1+\eta}\right\},
\ee 
which is,
\begin{equation}
\begin{split}
P_E(\eta)= &
\frac{1}{8(1+\eta)}Tr\bigg(M_{e=0}^{E=0}(\rho_{00}+\rho_{01})+ \\ & M_{e=0}^{E=1}(\rho_{10}+\rho_{11})\bigg)
+ \\& \frac{\eta}{8(1+\eta)} Tr\bigg(M_{e=0}^{E=0}(\rho_{00}+\rho_{10})+ \\ & M_{e=0}^{E=1}(\rho_{01}+\rho_{11})\bigg).
\end{split}
\end{equation}
After plugging in (\ref{c}),(\ref{a4}), this yields,
\begin{equation}
\begin{split}
P_E^{max}(\eta)& =\max_{\alpha_0}(P_E(\eta)) \\ & = \frac{1}{4}\left(2+\cos \alpha_0+\frac{1-\eta}{1+\eta}\sin\alpha_0\right).
\end{split}
\end{equation}
This expression is maximized for $\alpha_\eta=\arctan\left(\frac{1-\eta}{1+\eta}\right)$. Hence, we obtain the security condition (\ref{m1}).
\subsection*{(iii)}
Here the success probability for Eve is,
\begin{equation}
P_E(\eta)= \frac{1}{8(1+\eta)}Tr \sum_{a_0a_1eb} \eta^{1-\delta_{eb}}P(E=a_b|a_0,a_1,e,b),
\end{equation}
which can in-turn be expressed as,
\begin{equation}
\begin{split}
P_E(\eta)= \frac{1}{8} Tr\bigg[ & \sum_{i,j} \tilde{\rho}_{i,i,j} \frac{M_{i,i}^{E=j} \otimes \mathbb{I} + \eta M_{i,1-i}^{E=j} \otimes \mathbb{I} }{1 + \eta} + \\ & \tilde{\rho}_{i,1-i,j} \frac{M_{i,i}^{E=1-\delta_{i,j}} \otimes \mathbb{I} + \eta M_{i,1-i}^{E=\delta_{i,j}} \otimes \mathbb{I} }{1 + \eta} \bigg].
\end{split}
\end{equation}
Notice that this expression constitutes four independent elements  for specific value of the pair $(i,j)$. In order to find $P_E(\eta)^{max}$ we need only find the maximizing condition for one term. Let's consider a particular pair $(i,j)$, then the expression for $P_E(\eta)^{max}$ simplifies to,
\begin{equation}
\begin{split}
P_E^{max}(\eta)= & \frac{1}{2} \max \bigg\{ Tr\bigg(  \tilde{\rho}_{i,i,j} \frac{M_{i,i}^{E=j} \otimes \mathbb{I} + \eta M_{i,1-i}^{E=j} \otimes \mathbb{I} }{1 + \eta} + \\ & \tilde{\rho}_{i,1-i,j} \frac{M_{i,i}^{E=1-\delta_{i,j}} \otimes \mathbb{I} + \eta M_{i,1-i}^{E=\delta_{i,j}} \otimes \mathbb{I} }{1 + \eta} \bigg) \bigg\}.
\end{split}
\end{equation}
This equation is maximized when $M_{i,i}^{E=j}=M_{i,1-i}^{E=j}$ or $M_{i,i}^{E=1-\delta_{i,j}}=M_{i,1-i}^{E=\delta_{i,j}}$ yielding the same security condition as (\ref{maxMe}).
\subsection*{(iv)}
In the case when Eve does not have access to quantum memory (IR) we can re-write (\ref{max3}) in a convenient way as,
\begin{equation}
P_E^{max}(\eta)=\frac{1}{4}\max \bigg\{P_E^{a_0=a_1=a_2} + P_E^{NOT\{a_0=a_1=a_2\}} \bigg\},
\end{equation}
where $P_E^{a_0=a_1=a_2}$ is Eve's success probability for the case when all three of the input bits of Alice are equal and $P_E^{NOT\{a_0=a_1=a_2\} }$ is Eve's success probability for the case when the three inputs of Alice are not equal. As Alice's states and Bob's measurement that maximize $P_E^{a_0=a_1=a_2}$ remain the same irrespective of the fact whether Eve was able to guess Bob's input correctly or not we can further re-write this as,
\begin{equation}
P_E^{max}(\eta)=\frac{1}{4}\bigg( 1 + \max \bigg\{P_E^{NOT\{a_0=a_1=a_2\}} \bigg\} \bigg).
\end{equation}
Following exactly the same steps as above this yields the security condition (\ref{3to1max}).
\subsection*{(v)}
We find that the optimal states are,
\begin{eqnarray}
&& |000\rangle=|0\rangle,\nonumber\\
&& |001\rangle=\cos \frac{\alpha}{2}|0\rangle+\sin\frac{\alpha}{2}|1\rangle,\nonumber\\
&& |010\rangle=\cos \frac{\alpha}{2}|0\rangle+e^{i\beta}\sin\frac{\alpha}{2}|1\rangle,\nonumber\\
&& |100\rangle=\cos \frac{\alpha}{2}|0\rangle+e^{-i\beta}\sin\frac{\alpha}{2}|1\rangle,\nonumber\\
&& |111\rangle=|000\rangle^\perp ,\nonumber \\
&& |110\rangle=|001\rangle^\perp ,\nonumber \\
&& |101\rangle=|010\rangle^\perp ,\nonumber \\
&& |011\rangle=|100\rangle^\perp,\label{a11}
\end{eqnarray}
where $\alpha$ and $\beta$ are parameters controlled by Eve.  In this case optimal encoding for standard $(3\to1)$ QRAC is reproduced for $\alpha=\arccos\frac{1}{3}$ and $\beta=2\pi/3$.
For Eve's measurements ($ M_e^{E=a_b}=|M_e^{E=a_b}\rangle\langle M_e^{E=a_b}|$) we use the following parametrization
\begin{eqnarray}
&& |M_e^{E=0}\rangle = \cos\frac{\alpha_e}{2}|0\rangle+e^{i\beta_e}\sin\frac{\alpha_e}{2}|1\rangle \nonumber \\
&& |M_e^{E=1}\rangle = |M_e^{E=0}\rangle^\perp \label{a2},
\end{eqnarray} 
A straightforward maximization yields the security condition (\ref{max3to1res}).